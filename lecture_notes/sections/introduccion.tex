\section{Introducción}
\subsection{Motivación: Desafíos de la inferencia cosmológica moderna}
La cosmología observacional enfrenta una coyuntura crítica impulsada por la explosión en la cantidad y complejidad de los datos disponibles. Experimentos recientes como \textit{Planck} \citep{Planck2018} y Simons Observatory \citep{SO2019}, junto con proyectos futuros como CMB-S4 \citep{CMB-S42016}, generan conjuntos de datos cuyo análisis exhaustivo mediante métodos tradicionales basados en cadenas de Markov (MCMC) requiere recursos computacionales prohibitivos. Un caso emblemático es el análisis de los datos de \textit{Planck} 2018, que necesitó más de 10 millones de muestras MCMC para alcanzar convergencia en espacios de parámetros de más de 30 dimensiones \citep{Planck2018}, con un coste escalando como $\mathcal{O}(N^3)$ para $N$ modos observados \citep{Schneider2020}. Esta barrera computacional se agudizará en la próxima generación de experimentos con resoluciones angulares sub-arcminuto.

Paralelamente, la reducción de errores estadísticos por debajo del 1\% en mediciones de polarización \citep{BICEP2021} ha expuesto limitaciones fundamentales en el paradigma de inferencia actual. Errores sistemáticos anteriormente despreciables, tanto instrumentales (ruido correlacionado, efectos de haz asimétrico) como teóricos (no-Gaussianidad primordial, efectos de segundo orden en transferencia radiativa \citep{Challinor2011}), dominan ahora las incertidumbres. Particularmente problemática resulta la creciente evidencia de desviaciones de Gaussianidad en pequeños escalones angulares ($\ell > 2000$), donde los efectos de lente gravitacional distorsionan significativamente las estadísticas del fondo cósmico de microondas \citep{Planck2018NL}. Estas complejidades se amplifican al considerar análisis conjuntos de múltiples estadísticos como $C_\ell^{TT}$, $C_\ell^{TE}$ y funciones de correlación $\xi_+(\theta)$, cuyas covarianzas cruzadas introducen no-linearidades no triviales \citep{EFS2004}.

Este panorama exige desarrollar frameworks de inferencia alternativos que superen tres limitaciones clave: primero, la incapacidad de los métodos basados en likelihood exacto para escalar a los volúmenes de datos futuros; segundo, la dependencia crítica de aproximaciones Gaussianas cada vez más cuestionables; y tercero, la dificultad para incorporar sistemáticos complejos en modelos analíticos. La inferencia basada en simulaciones (SBI) emerge como solución prometedora al integrar emuladores de alta dimensión \citep{Alsing2019}, técnicas de compresión óptima de datos \citep{Jeffrey2021} y algoritmos de muestreo eficiente \citep{Papamakarios2019}, permitiendo una caracterización rigurosa de parámetros cosmológicos directamente desde observaciones sin necesidad de likelihoods explícitos.

