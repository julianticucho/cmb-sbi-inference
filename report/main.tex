% Template:     Artículo LaTeX
% Documento:    Archivo principal
% Versión:      1.3.7 (23/08/2024)
% Codificación: UTF-8
%
% Autor: Pablo Pizarro R.
%        pablo@ppizarror.com
%
% Manual template: [https://latex.ppizarror.com/articulo]
% Licencia MIT:    [https://opensource.org/licenses/MIT]

% CREACIÓN DEL DOCUMENTO
\documentclass[
	spanish, % Idioma: spanish, english, etc.
	letterpaper, oneside
]{article}

% INFORMACIÓN DEL DOCUMENTO
\def\documenttitle {Research Report AS4103}
\def\documentsubtitle {}
\def\documentdate {\today}
\def\journalname {:v}

% IMPORTACIÓN DEL TEMPLATE
\input{template}

% CONFIGURA LOS AUTORES
\def\authorshf {Julián Proboste} % Nombres en header-footer

% Otras funciones:
%	\addauthornonumber{Nombre autor}
%	\addauthorsamenumber[ORCID]{Nombre autor}{Institución}{email}
\addauthor{Julián Proboste}{Student of Astronomy - Computer Science, Universidad de Chile}{julian.proboste@ug.uchile.com}

\setlength{\parindent}{0pt}

% INICIO DE PÁGINAS
\begin{document}

% CONFIGURACIÓN DE PÁGINA Y ENCABEZADOS
\templatePagecfg

% CONFIGURACIONES FINALES
\templateFinalcfg

% ======================= INICIO DEL DOCUMENTO =======================

% Título y nombre del autor
\inserttitle

\renewcommand{\nameabstract}{Abstract}
\renewcommand{\nameltcont}{Table of Contents}
\renewcommand{\nameltwfigure}{Figure}
\renewcommand{\namereferences}{References}

\begin{abstract}
This work addresses the estimation of cosmological parameters from Cosmic Microwave Background (CMB) power spectra using simulation-based inference (SBI) methods. Unlike the traditional approach, which relies on Gaussian assumptions for the likelihood function, we employ modern inference techniques such as Sequential Neural Posterior Estimation (SNPE) and Neural Posterior Score Estimation (NPSE). Between 25,000 and 100,000 simulations of the power spectra $C_{\ell}^{TT}$, $C_{\ell}^{EE}$, $C_{\ell}^{BB}$, and $C_{\ell}^{TE}$ were generated using the \texttt{CAMB} code, considering only scalar modes. The results show that NPSE provides wider but less biased posterior distributions compared to SNPE, thus improving the robustness of the inference. Furthermore, we analyze the well-known degeneracy between the optical depth to reionization $\tau$ and the primordial amplitude $A_s$, showing that the inclusion of low-$\ell$ polarization data is crucial to properly constrain these parameters. Overall, this study demonstrates the potential of SBI methods as a flexible and efficient alternative to traditional Bayesian approaches in cosmology.
\end{abstract}
% Contenido
\begin{multicols}{2}
	\section{Introduction}
\subsection{Cosmological Context}
The discovery of the Cosmic Microwave Background (CMB) by Arno Penzias and Robert Wilson~\cite{penzias} in 1965 confirmed the existence of isotropic microwave radiation permeating the universe. Subsequent experiments showed that this radiation follows a blackbody spectrum with a temperature of \( T = 2.7255 \, \text{K} \) \cite{ryden}. The energy density of the CMB is low, with approximately 411 photons per cubic centimeter, and its characteristic wavelength (\(\sim 2\,\mathrm{mm}\)) places it in the microwave region of the electromagnetic spectrum~\cite{ryden}.

The existence of the CMB constituted key evidence in favor of the Big Bang model over the Steady State model. In an expanding universe that was initially hot and dense, matter was ionized, making the universe opaque. As the universe expanded and cooled, electrons and protons combined to form neutral atoms, allowing photons to decouple from matter. This radiation, which originally had a temperature of \(\sim 2970 \, \text{K}\), has cooled to the current \(2.7255 \, \text{K}\) due to cosmic expansion \cite{ryden}.

With the arrival of the \textit{WMAP} and \textit{Planck} probes, high-precision measurements of CMB anisotropies were achieved, which are fundamental for understanding the primordial inhomogeneities that gave rise to the large-scale structure of the universe. Combining these results with independent observations, such as type Ia supernovae and galaxy distribution, has allowed for refined estimates of cosmological parameters, consolidating the \(\Lambda\)CDM model as the standard framework of modern cosmology.

\subsection{Traditional Inference}
In modern cosmology, CMB analysis follows a well-established methodology, which now faces new challenges due to the volume and complexity of current data. The starting point is the temperature and polarization maps obtained by missions such as Planck~\cite{planck2018} and, in the near future, the Simons Observatory~\cite{simons2019}. These maps contain primordial information about both cosmological parameters and the initial conditions of the universe.

Traditional analysis takes these maps and extracts the power spectra \(C_{\ell}^{TT}\), \(C_{\ell}^{EE}\), \(C_{\ell}^{TE}\), and \(C_{\ell}^{BB}\), which quantify angular correlations at different scales (\(\ell \sim 180^\circ/\theta\)). This compression preserves Gaussian information and reduces data dimensionality. Mathematically, the power spectra correspond to the coefficients in the Legendre polynomial expansion of the two-point correlation function:

\begin{equation}
C(\theta)=\sum_{\ell=0}^{\infty}\frac{2\ell+1}{4\pi}C_{\ell}P_{\ell}(\cos \theta)
\end{equation}

where the two-point correlation function \(C(\theta)\) is defined as:

\begin{equation}
C(\theta)=\langle T(\hat{n}_1)T(\hat{n}_2) \rangle
\end{equation}

with \(T(\hat{n})\) being the temperature field in the unit direction \(\hat{n}\). The inference of cosmological parameters is carried out through likelihood functions, usually Gaussian, that compare observed spectra with theoretical predictions. Bayesian inference combines these likelihoods with prior distributions incorporating theoretical knowledge and independent constraints. In practice, exploration of the parameter space is typically performed using Markov Chain Monte Carlo (MCMC) methods, which sample the posterior distribution in high-dimensional spaces.

This approach has allowed precise determination of the \(\Lambda\)CDM model parameters but requires assuming a likelihood function that is not always exact. Moreover, the Gaussian likelihood assumption may be too restrictive. In the following section, an alternative simulation-based approach is presented, which allows inference in situations where the likelihood is intractable or unknown.

\subsection{Simulation-Based Inference}
The traditional approach presents two fundamental limitations: the loss of non-Gaussian information during compression to power spectra, and the dependence on Gaussian assumptions in the likelihood. Both problems are closely related, since attempting inference with non-Gaussian summary statistics while assuming a Gaussian likelihood can lead to biased results. Simulation-based inference (SBI) aims to perform statistical inference in situations where the likelihood function is intractable or unknown.

Hybrid methods such as ABC (\textit{Approximate Bayesian Computation}) address this problem by directly comparing simulated data with observed data using a distance metric. In this approach, multiple simulations are generated from different parameter values \(\theta\) and those for which the simulated data \(x\) sufficiently resemble the observed data, according to a tolerance threshold, are retained. This method strongly depends on the choice of summary statistics and may require a very large number of simulations to obtain a reasonable approximation of the posterior distribution.

More recently, modern machine learning-based techniques such as SNPE (\textit{Sequential Neural Posterior Estimation}) \cite{SNPE_C} or NPSE (\textit{Neural Posterior Score Estimation}) \cite{NPSE_1} \cite{NPSE_2} have revolutionized this approach. Instead of directly comparing data, these methods reformulate the problem as one of density estimation: the joint distribution of simulated pairs \((\theta, x)\) is modeled, allowing approximation of the posterior distribution \(p(\theta|x)\). Tools such as deep neural networks and normalizing flows enable training expressive models that learn directly the relationship between data and parameters from synthetic samples, avoiding the explicit computation of the likelihood and making inference much more efficient.

\subsection{Project Objective}
The main objective of this project is the implementation of simulation-based inference techniques to estimate cosmological parameters from angular power spectra of the CMB. In the methodology section, the problem is introduced and the proposed methodology is described, while the results section presents the outcomes obtained using the proposed approach.



	\section{Methodology}
\subsection{Cosmological Parameters}
We considered the parameters of the $\Lambda$CDM model introduced in the Markov chains by the \textit{Planck Collaboration 2018} \cite{planck2018}, namely the reduced density contrast of cold dark matter $\omega_c$ and baryons $\omega_b$, the angular scale of the sound horizon at recombination $100\theta_{MC}$, the amplitude of the primordial spectrum $A_s$, the spectral index $n_s$, and the optical depth to reionization $\tau$. Initially, only the first five cosmological parameters were used, leaving out the optical depth $\tau$. This was due to its well-known temperature degeneracy with the parameter $A_s$.  

\subsection{Simulations}
Since a large number of power spectra need to be simulated by varying the cosmological parameters, it is necessary to adopt a probability distribution from which to initially draw these parameters, known as the \textit{prior}. In this project, a flat prior (uniform distribution) was used within a range of $\pm 10$ sigmas around the fiducial values reported by the \textit{Planck Collaboration 2018} \cite{planck2018}.  

Between 25,000 and 100,000 random samples were drawn from the prior, and for each cosmology a power spectrum of each type ($C_{\ell}^{TT}$, $C_{\ell}^{EE}$, $C_{\ell}^{BB}$, $C_{\ell}^{TE}$) was simulated in the multipole range $0 \leq \ell \leq 2500$. The simulations included only scalar modes (\textit{tensor-to-scalar ratio} \(r = 0\)) and accounted for lensing effects. These simulations were carried out using the \texttt{CAMB} framework in Python \cite{CAMB}. Figure \ref{fig:example_sim} shows three examples for each simulated power spectrum.  

\begin{figure}
    \centering
    \includegraphics[scale=0.225]{img/cmb_aps_subplot_0.pdf}
    \caption{Example of three simulations of the power spectra $C_{\ell}^{TT}$, $C_{\ell}^{EE}$, $C_{\ell}^{BB}$, $C_{\ell}^{TE}$, respectively. Each simulation was generated from a random sample of the prior.}
    \label{fig:example_sim}
\end{figure}

\subsection{Inference Models}
To estimate the posterior distribution of the cosmological parameters, different inference models were used. In particular, \textit{Sequential Neural Posterior Estimation} \cite{SNPE_C} (SNPE) and \textit{Neural Posterior Score Estimation} (NPSE) \cite{NPSE_1} \cite{NPSE_2} were employed, implemented in the \textit{SBI} Python framework \cite{SBI}. Both NPSE and SNPE are Bayesian inference methods for simulation-based models, but they differ in their representation of the posterior and their training mechanisms. NPSE uses conditional diffusion models and \textit{score matching} to estimate density gradients, thereby avoiding the use of normalizable models \cite{NPSE_1}, while SNPE-C employs \textit{normalizing flows} to directly model the posterior and requires importance-weighting corrections \cite{SNPE_C}. In the specific case of this project, NPSE showed better adaptation to the data, producing wider but less biased posteriors than SNPE, as shown in Figure \ref{fig:model_comparison}.  

\begin{figure}
    \centering
    \includegraphics[scale=0.35]{img/inference_model_comparison.pdf}
    \caption{PPC diagnostic of the results obtained with the inference models SNPE and NPSE. Both models were trained with the same 100,000 simulations of the $C_{\ell}^{TT}$ power spectrum. The dashed line indicates the true value of the parameters. The NPSE model yields wider but less biased distributions compared to SNPE.}
    \label{fig:model_comparison}
\end{figure}

\subsection{Posterior Predictive Check (PPC)}
The result of training the selected inference model is known as a density estimator, which is capable of generating random samples from the trained posterior distribution given an observation. To analyze and evaluate the quality of the model, the \textit{Posterior Predictive Check} (PPC) \cite{SBI} method was used. This consists of selecting a true value for the cosmological parameters and simulating an observed power spectrum, and then sampling the previously trained posterior distribution from this observation. In this way, the posterior distribution obtained by the inference model can be compared with the true parameter value, typically represented by a dashed line, as shown in Figure \ref{fig:model_comparison}.




	\section{Results}
\subsection{Inference without reionization parameter}
Since the parameter $\tau$ has a known physical degeneracy with the parameter $A_s$ \cite{HuWhite1997}, the set of six parameters was temporarily reduced to five, with $\tau$ later added back. The results of training NPSE inference models with the power spectra $C_{\ell}^{TT}$, $C_{\ell}^{EE}$, $C_{\ell}^{BB}$, and $C_{\ell}^{TE}$ in the range $0 \leq \ell \leq 2500$ are shown in Figure \ref{fig:parameters_inference}.  

\begin{figure}
    \centering
    \includegraphics[scale=0.35]{img/data_comparison.pdf}
    \caption{NPSE inference models trained with 100,000 simulations of the power spectra $C_{\ell}^{TT}$, $C_{\ell}^{EE}$, $C_{\ell}^{BB}$, and $C_{\ell}^{TE}$. Each color represents a model trained with a different power spectrum. The dashed line shows the true value of the parameters.}
    \label{fig:parameters_inference}
\end{figure}

The results show how the choice of dataset ($C_{\ell}^{TT}$, $C_{\ell}^{EE}$, $C_{\ell}^{BB}$, $C_{\ell}^{TE}$) affects the posterior distributions of the selected cosmological parameters. The variation in $\omega_b$, $\omega_c$, $n_s$, $\ln(10^{10}A_s)$, and $100\theta_{MC}$ is small but non-negligible, reflecting the sensitivity of each parameter to the type of information included. In particular, the temperature and E-mode spectra (TT, EE, TE) introduce slight tensions compared to the B-mode spectrum (BB), highlighting the lack of information contained in the B-mode spectra computed only from scalar perturbations, without including primordial B-modes. Figure \ref{fig:pred_vs_obs} shows the comparison between the power spectra corresponding to the true parameters and those corresponding to a random sample from the posterior distribution obtained by training an NPSE inference model with 100,000 simulations of the $C_{\ell}^{TT}$ power spectrum.  

\subsection{Inference with reionization parameter}
When the parameter $\tau$ is included, the full set of six cosmological parameters is recovered. Figure \ref{fig:parameters_inference_tau} shows the results obtained by training the NPSE inference model with 100,000 simulations including $\tau$. This parameter introduces a clear degeneracy with $\ln(10^{10}A_s)$, which results in a broadening and a slight bias in the posterior distribution of both parameters compared to the case without $\tau$, as well as an elongation of their joint distribution.  

Moreover, the inclusion of $\tau$ produces slight variations in the posterior distributions of $n_s$ and $\omega_c$, although these remain within statistical dispersion. The temperature (TT) and polarization (EE, TE) spectra, particularly at large scales, provide the strongest sensitivity to the value of $\tau$, while the TT modes at high multipoles are unable to break the degeneracy. Consequently, this confirms the necessity of incorporating information from the first 30 multipoles of the polarization spectra (lowEE, lowTE) to improve the precision in the estimation of $\tau$ and, therefore, of $A_s$.

\begin{figure}
    \centering
    \includegraphics[scale=0.3]{img/data_comparison_tau_100000_1.pdf}
    \caption{Results of NPSE inference considering the six standard cosmological parameters, including $\tau$. The degeneracy between $\tau$ and $\ln(10^{10}A_s)$ is evident, producing a broadening of their posterior distributions. The different datasets (TT, TE, EE, and TT+lowEE+lowTE) highlight the importance of including low-multipole information to constrain the value of $\tau$.}
    \label{fig:parameters_inference_tau}
\end{figure}

\subsection{Adding Noise}
To approximate more realistic observational conditions, noise is introduced into the simulations of the power spectra $C_{\ell}^{TT}$. We mainly consider two sources: the instrumental noise of the experiment and the partial sky coverage, which adds additional cosmic variance. First, instrumental noise is incorporated into the theoretical spectra using a model based on the angular resolution of the instrument (\(\theta_{\text{fwhm}}\)) and the pixel sensitivity (\(\sigma_T\)) \cite{noiseCole}. The instrumental noise term \(N_\ell^{\mathrm{TT}}\), which is added to the theoretical power spectrum \(C_\ell\), is defined as:

\begin{equation}
N_\ell^{\mathrm{TT}} = \left(\theta_{\text{fwhm}} \cdot \sigma_T \right)^2 \exp\left[ \ell (\ell + 1) \frac{\theta_{\text{fwhm}}^2}{8 \ln 2} \right],
\end{equation}

where \(\theta_{\text{fwhm}}\) is expressed in radians. This expression models the sky smoothing due to the finite resolution of the instrument.

Subsequently, the observation of a partial sky is simulated, considering that only a fraction \(f_{\text{sky}} < 1\) of the full sky is measured. As a result, the observable estimator of \(C_\ell\), denoted \(\hat{C}_\ell\), becomes a random variable whose dispersion depends on \(f_{\text{sky}}\). For low multipoles (\(\ell < \ell_{\text{transition}}\)), this variance is modeled through a scaled chi-squared distribution:

\begin{equation}
\hat{C}_\ell = \frac{1}{\nu_\ell} \sum_{i=1}^{\nu_\ell} X_i^2, \quad X_i \sim \mathcal{N}(0, \sqrt{C_\ell}),
\end{equation}

where \(\nu_\ell = \text{round}(f_{\text{sky}} \cdot (2\ell + 1))\) represents the effective number of degrees of freedom. This formulation correctly captures the statistical dispersion of the estimator when the number of available modes is limited. For high multipoles (\(\ell \geq \ell_{\text{transition}}\)), it is assumed that the estimator can be approximated by a normal distribution centered on \(C_\ell\) with variance:

\begin{equation}
\hat{C}_\ell \sim \mathcal{N}\left(C_\ell, \frac{2 C_\ell^2}{f_{\text{sky}} (2\ell + 1)} \right).
\end{equation}

This Gaussian approximation is valid thanks to the central limit theorem, since in this regime the number of available modes is sufficiently large.

\begin{figure}
    \centering
    \includegraphics[scale=0.41]{img/grafico.pdf}
    \caption{Multiple realizations of a power spectrum for the same combination of cosmological parameters, including instrumental noise and the effect of partial sky coverage. Each color represents a different realization.}
    \label{fig:realizations_noise}
\end{figure}

To capture the randomness introduced by noise in the power spectra, multiple realizations of each combination of cosmological parameters were generated in the training set \cite{novaes}. Each cosmology was simulated several times, applying instrumental noise and the variance associated with partial sky coverage, so that the model can learn the natural dispersion of the observables due to these sources of uncertainty. This approach allows the inference model not only to learn the average values of the spectra, but also to internalize the statistical variability caused by noise. In this way, the posterior predictions more realistically reflect the uncertainty expected in real observational data.

Figure \ref{fig:realizations_noise} shows an example of multiple realizations of the power spectrum for the same cosmology, where each colored line represents a different realization. It can be observed how noise produces significant variations in the spectrum, especially at scales where instrumental noise and partial sky coverage have the greatest effect.

\begin{figure}
    \centering
    \includegraphics[scale=0.35]{img/01_repeat_noise_comparison_100000.pdf}
    \caption{Inference of cosmological parameters from the TT power spectrum. The model training without additional realizations (\texttt{TT}) is compared to the case with five noisy realizations per cosmology (\texttt{TT repeat5}).}
    \label{fig:inference_with_noise}
\end{figure}

Figure \ref{fig:inference_with_noise} presents the posterior distributions of cosmological parameters obtained under two training schemes. In the case without additional realizations (blue plot), the distributions appear narrower, reflecting an underestimation of the true uncertainty associated with the data. In contrast, when five noisy realizations per cosmology are included in the training (green plot), the posteriors are more consistently centered and capture more faithfully the dispersion induced by instrumental noise.

Although the scheme with realizations still shows residual artificial degeneracies and biases, the results are significantly improved compared to the scheme without realizations. Most parameters exhibit intervals that are closer to the expected statistical variability, indicating that the model, when exposed to multiple realizations, better internalizes the stochastic nature of the data and avoids an overconfident characterization of the posteriors.










	\input{sections/conclusion.tex}
	\section{Appendix}
\subsection{Figures}
\begin{figure}
    \centering
    \includegraphics[scale=0.35]{img/nsims_comparison.pdf}
    \caption{PPC diagnostic of the inference performed with 25,000 and 100,000 training simulations of the $C_{\ell}^{TT}$ power spectrum. Both models were trained with the NPSE architecture. The dashed line shows the true value of the parameters. The model trained with 100,000 simulations exhibits better convergence than the one trained with 25,000 simulations, with a more concentrated distribution and smoother boundaries.}
    \label{fig:nsims_comparison}
\end{figure}

\begin{figure}
    \centering
    \includegraphics[scale=0.45]{img/NPSE_TT_tau_100000}
    \caption{Evolution of the loss function on training and validation sets across 500 epochs. The NPSE model was trained with 100,000 simulations of the $C_{\ell}^{TT}$ spectrum.}
    \label{fig:nsims_comparison}
\end{figure}

\begin{figure}
    \centering
    \includegraphics[scale=0.225]{img/cmb_aps_pred_vs_obs_0.pdf}
    \caption{Comparison of the power spectra $C_{\ell}^{TT}$, $C_{\ell}^{EE}$, $C_{\ell}^{BB}$, and $C_{\ell}^{TE}$ simulated with the true parameter values and the power spectra obtained from a random posterior sample derived from training an NPSE inference model with 100,000 simulations of the $C_{\ell}^{TT}$ power spectrum. The lower panel of each subplot shows the difference between the observed and predicted power spectra.} 
    \label{fig:pred_vs_obs}
\end{figure}

\begin{figure}
    \centering
    \includegraphics[scale=0.35]{img/inference_with_noise.pdf}
    \caption{Comparison of cosmological inference from the TT power spectrum, considering two training schemes: with unbinned noise (\texttt{TT+noise}) 
    and with noise averaged over 500 bins (\texttt{TT+noise+binned}). These NPSE models were trained with 100,000 base cosmologies}
    \label{fig:nsims_comparison}
\end{figure}

\subsection{Simulation-based Inference}
Un desafío persistente en la inferencia bayesiana es el tratamiento de simuladores estocásticos complejos cuyas funciones de verosimilitud son intrínsecamente intratables. Recientemente, los métodos de Inferencia Basada en Simulación (Simulation-Based Inference, SBI) que utilizan estimadores de densidad condicional basados en redes neuronales han surgido como una solución prometedora. Estas técnicas aprenden la distribución posterior a partir de simulaciones propuestas de manera adaptativa, superando las limitaciones de escalabilidad de los enfoques clásicos, como la Computación Bayesiana Aproximada (ABC), en problemas de alta dimensión.

Dentro del aprendizaje automático, las técnicas SBI se pueden clasificar en dos categorías principales. Los métodos de \textit{Verosimilitud Sintética} (Sequential Neural Likelihood, SNL) se centran en estimar la verosimilitud $p(x|\theta)$ para luego obtener la posterior mediante procedimientos de inferencia adicionales (e.g., MCMC). Por otro lado, los métodos de \textit{Estimación de la Posterior} (Sequential Neural Posterior Estimation, SNPE) apuntan directamente a aprender la distribución posterior $p(\theta|x)$. Si bien este último enfoque permite una inferencia amortizada y aprovecha la capacidad de las redes neuronales para aprender características directamente de los datos, se enfrenta a una limitación fundamental: el uso de distribuciones de propuesta distintas a la previa requiere la aplicación de correcciones post-hoc numéricamente inestables o el uso de ponderación de importancia, lo que puede aumentar la varianza durante el entrenamiento y limitar el rendimiento práctico.

\subsection{Automatic Posterior Transformation}
Para abordar estas limitaciones, \cite{SNPE_C} introdujeron la Transformación Automática de la Posterior (APT o SNPE-C). APT es un método de estimación posterior neuronal secuencial que combina las ventajas de ambos enfoques. Al replantear el problema de inferencia como uno de estimación de razón de densidades, APT permite el uso de distribuciones de propuesta arbitrarias y actualizadas dinámicamente sin necesidad de pesos de importancia o correcciones inestables. Esta flexibilidad, combinada con la compatibilidad con potentes estimadores de densidad basados en flujos normales (Normalizing Flows), hace de APT un método más escalable y eficiente, capaz de operar directamente sobre datos complejos como series temporales multidimensionales o imágenes.

El problema central de la Inferencia Basada en Simulación consiste en estimar la distribución posterior $p(\theta|x_o)$ de los parámetros $\theta$ dados unos datos observados $x_o$, cuando el simulador define implícitamente un modelo con verosimilitud $p(x|\theta)$ analíticamente intratable. Este marco se puede abordar como un problema de estimación de densidad condicional, donde una red neuronal $F$ con parámetros $\phi$ aprende un mapeo desde los datos $x$ hasta los parámetros $\psi$ de una distribución $q_\psi(\theta)$ que aproxima la posterior $p(\theta|x)$. El entrenamiento se realiza minimizando la pérdida

\begin{equation}
\label{eq:loss}
\mathcal{L}(\phi) = - \sum_{j=1}^N \log q_{F(x_j, \phi)}(\theta_j),
\end{equation}

sobre un conjunto de simulaciones $\{(\theta_j, x_j)\}$ generadas a partir de la distribución previa $p(\theta)$. Dado que el interés final reside en la posterior condicionada a $x_o$, resulta ineficiente seguir muestreando parámetros de la previa inicial una vez se dispone de una estimación preliminar de $p(\theta|x_o)$. Los métodos secuenciales abordan esto refinando iterativamente la distribución de propuesta $\tilde{p}(\theta)$ para concentrar las simulaciones en regiones del espacio de parámetros más relevantes para $x_o$. No obstante, este enfoque introduce una dificultad: si el entrenamiento se realiza con parámetros muestreados de $\tilde{p}(\theta) \neq p(\theta)$, el estimador converge a la posterior de propuesta $\tilde{p}(\theta|x)$, que está relacionada con la posterior verdadera mediante

\begin{equation}
\label{eq:posteriorr}
\tilde{p}(\theta|x) = \frac{\tilde{p}(\theta)p(x)}{p(\theta)\tilde{p}(x)}.
\end{equation}

Existen varias estrategias para resolver este problema. \textbf{SNPE-A} \cite{} restringe las familias de distribuciones (usualmente mezclas de Gaussianas) y las propuestas (Gaussianas) para permitir una corrección analítica \textit{post-hoc}, pero esta falta de flexibilidad limita su aplicabilidad. \textbf{SNPE-B} \citep{} utiliza ponderación de importancia en la función de pérdida para recuperar directamente $p(\theta|x)$, eliminando las restricciones pero introduciendo una alta varianza en las actualizaciones del gradiente que puede ralentizar o desestabilizar el entrenamiento. Una alternativa diferente es \textbf{SNL} \citep{}, que estima la verosimilitud $p(x|\theta)$ en lugar de la posterior, lo que permite usar cualquier propuesta pero requiere una etapa adicional de muestreo MCMC para obtener la posterior, incrementando el coste computacional para distribuciones complejas. 

La Transformación Automática de la Posterior es una técnica que combina las propiedades deseables de los enfoques de estimación de la posterior y de los basados en verosimilitud. APT aprende a inferir la posterior verdadera maximizando una posterior de propuesta estimada. Utiliza la relación \ref{eq:posteriorr} para formar una parametrización que permite transformar automáticamente entre estimaciones de $p(\theta|x)$ y $\tilde{p}(\theta|x)$, facilitando así leer directamente la estimación de la posterior. Este enfoque evita los desafíos numéricos de las técnicas SNPE anteriores y permite utilizar una amplia gama de distribuciones de propuesta y estimadores de densidad. En APT, $q_{F(x,\phi)}(\theta)$ representa una estimación de $p(\theta|x)$. Para transformarla en una estimación de $\tilde{p}(\theta|x)$, se define:

\begin{equation}
\tilde{q}_{x,\phi}(\theta) = q_{F(x,\phi)}(\theta) \frac{\tilde{p}(\theta)}{p(\theta)} \frac{1}{Z(x, \phi)}
\end{equation}

donde 

\begin{equation}
Z(x, \phi) = \int_\theta q_{F(x,\phi)}(\theta) \frac{\tilde{p}(\theta)}{p(\theta)}
\end{equation}

es una constante de normalización. La red se entrena minimizando la pérdida de (\ref{eq:loss}). Si el estimador de densidad condicional es suficientemente expresivo, minimizar $\tilde{\mathcal{L}}(\phi)$ garantiza que $q_{F(x,\phi)}(\theta) \to p(\theta|x)$ y $\tilde{q}_{x,\phi}(\theta) \to \tilde{p}(\theta|x)$ cuando $N \to \infty$. Al igual que otros métodos secuenciales, APT refina iterativamente los pesos de la red $\phi$ y la propuesta $\tilde{p}(\theta)$ a lo largo de múltiples rondas de simulación. Una ventaja clave es que puede entrenarse con datos de todas las rondas simplemente sumando sus términos de pérdida, a diferencia de SNPE-A (que no puede reutilizar datos entre rondas) o SNPE-B (que debe aplicar diferentes pesos de importancia).

\subsection{Neural Posterior Score Estimation}
Recientemente, \cite{NPSE_1} \cite{NPSE_2} introdujeron la Estimación Neuronal Secuencial del Puntaje de la Posterior (Sequential Neural Posterior Score Estimation, SNPSE), un método innovador para inferencia bayesiana en modelos basados en simuladores que se inspira en el notable éxito de los modelos basados en score en el modelado generativo. A diferencia de métodos anteriores como SNPE que se basan en estimadores de densidad normalizados (por ejemplo, flujos normales), SNPSE utiliza modelos de difusión condicionales basados en score para generar muestras de la distribución posterior de interés. El modelo se entrena mediante una función de objetivo que estima directamente el score (es decir, el gradiente del logaritmo de la densidad) de la posterior, aprovechando técnicas de \textit{score matching}.

Una ventaja fundamental de NPSE es que, al solo requerir estimaciones del gradiente del log-densidad, evita la necesidad de un modelo normalizable. Esto elimina restricciones arquitectónicas fuertes y la inestabilidad asociada con los objetivos de entrenamiento adversariales utilizados en otros métodos. El enfoque se puede implementar en una variante amortizada (NPSE) o embeberse en un procedimiento de entrenamiento secuencial (SNPSE), donde las simulaciones son guiadas por la aproximación actual de la posterior en la observación de interés, reduciendo significativamente el costo computacional de simulación.

El marco de Inferencia Basada en Simulación (SBI) con modelos de difusión aborda el problema clásico de inferir la distribución posterior $p(\theta|x_{\text{obs}})$ cuando solo se tiene acceso a un simulador que genera pares $(\theta, x) \sim p(\theta)p(x|\theta)$, pero la verosimilitud $p(x|\theta)$ es intrínsecamente intratable. La propuesta central utiliza modelos de difusión basados en puntajes (\textit{score-based}) condicionales \cite{} para muestrear directamente de la posterior. En este enfoque, se define un proceso directo de difusión (ecuación diferencial estocástica) que gradualmente agrega ruido a la distribución objetivo $p(\theta|x)$ hasta transformarla en una distribución de referencia manejable, típicamente una Gaussiana estándar.

\begin{equation}
d\theta_t = f(\theta_t, t)dt + g(t)dw_t
\end{equation}

La clave reside en que la reversión temporal de este proceso

\begin{equation}
d\bar{\theta}_t = \left[-f(\bar{\theta}_t, T-t) + g^2(T-t)\nabla_\theta \log p_{T-t}(\bar{\theta}_t|x)\right]dt + g(T-t)dw_t.
\end{equation}

es también un proceso de difusión, cuyas dinámicas dependen del puntaje $\nabla_\theta \log p_t(\theta_t|x)$ y pueden aproximarse mediante \textit{score matching}. Esto permite generar muestras de la distribución posterior de interés de manera eficiente y sin la necesidad de un modelo normalizable. Antes de abordar las variantes secuenciales, es crucial comprender el funcionamiento del método no secuencial (NPSE). La idea central de NPSE es entrenar una red neuronal $s_\psi(\theta_t, x, t)$ para que aproxime el \textbf{score} de la distribución posterior en diferentes niveles de ruido $t$. El puntaje se define como el gradiente del logaritmo de la densidad: $\nabla_{\theta} \log p_t(\theta_t | x)$. La training se realiza mediante la técnica de Denoising Score Matching (DSM) \cite{}. En DSM, se corrompen muestras $\theta_0 \sim p(\theta|x)$ con ruido gaussiano, obteniendo 

\begin{equation}
\theta_t \sim p_{t|0}(\theta_t|\theta_0) = \mathcal{N}(\theta_t; \theta_0, \sigma_t^2 I).
\end{equation}

La función de pérdida es:

\begin{equation}
\begin{aligned}
\mathcal{L}_{\text{DSM}}(\psi) 
= \frac{1}{2} \int_0^T \lambda_t 
\mathbb{E}_{p_{t|0}(\theta_t|\theta_0)p(\theta_0|x)}
\Big[ \| s_\psi(\theta_t, x, t) \\ 
- \nabla_{\theta_t} \log p_{t|0}(\theta_t|\theta_0) \|^2 \Big]\, dt
\end{aligned}
\end{equation}
donde $\lambda_t$ es una función de ponderación temporal. El término 

\begin{equation}
\nabla_{\theta_t} \log p_{t|0}(\theta_t|\theta_0) = -(\theta_t - \theta_0)/\sigma_t^2
\end{equation}

es el score del proceso de difusión directo, que actúa como target de training. Bajo condiciones de regularidad, minimizar esta pérdida equivale a hacer \textit{score matching} directo con el score verdadero $\nabla_{\theta} \log p_t(\theta_t | x)$. Una vez entrenada la red de scores, se puede generar muestras de la posterior $p(\theta|x_{\text{obs}})$ sustituyendo 

\begin{equation}
s_\psi(\theta_t, x_{\text{obs}}, t) \approx \nabla_{\theta} \log p_t(\theta_t|x_{\text{obs}})    
\end{equation}

en el proceso de reversión temporal o en la EDO de flujo de probabilidad. La principal ventaja de este enfoque es que no requiere que el modelo de densidad sea normalizable, evitando restricciones arquitectónicas y permitiendo el uso de poderosos modelos de difusión. El método secuencial preferido es SNPSE Truncado (TSNPSE), inspirado en \cite{}. En cada ronda $r$, la propuesta $\tilde{p}_r(\theta)$ se define como un promedio de versiones truncadas de la previa. Específicamente, 

\begin{equation}
\bar{p}_r(\theta) \propto p(\theta) \cdot \mathbb{I}\{\theta \in \text{HPR}_\epsilon(p^{r-1}_{\psi}(\theta|x_{\text{obs}}))\}.
\end{equation}

donde $\text{HPR}_\epsilon$ es la región de probabilidad más alta (con masa $1-\epsilon$) de la aproximación posterior de la ronda anterior. La propuesta final es 

\begin{equation}
\tilde{p}_r(\theta) = \frac{1}{r} \sum_{s=0}^{r-1} \bar{p}_s(\theta).
\end{equation}

La ventaja crucial de este enfoque es que, si la región de truncamiento contiene el soporte de la posterior verdadera $p(\theta|x_{\text{obs}})$, la propuesta es proporcional a la previa dentro de dicho soporte. Esto significa que la función de pérdida de \textit{score matching} sigue estando minimizada en el puntaje de la posterior verdadera, sin necesidad de aplicar correcciones por pesos de importancia, lo que conduce a un entrenamiento estable y eficaz. Se exploran análogos secuenciales de los métodos SNPE-A, SNPE-B y SNPE-C en el espacio del puntaje:

\begin{itemize}
    \item \textbf{SNPSE-A:} Realiza una corrección \textit{post-hoc} mediante remuestreo por importancia (SIR) de las muestras de la posterior de propuesta, usando los pesos $h_i = p(\tilde{\theta}_i) / \tilde{p}_r(\tilde{\theta}_i)$. Su precisión se ve limitada por la aproximación inherente en estos pesos cuando la posterior de propuesta aprendida no coincide bien con la verdadera.
    \item \textbf{SNPSE-B:} Incorpora pesos de importancia $p(\theta_0) / \tilde{p}_r(\theta_0)$ directamente en el objetivo de \textit{denoising score matching}. Si bien el minimizador teórico es el puntaje verdadero, los pesos de importancia de alta varianza often resultan en un entrenamiento inestable y un rendimiento pobre, un problema análogo al de SNPE-B.
    \item \textbf{SNPSE-C:} Propone una corrección automática en el espacio del puntaje. Estima el puntaje de la posterior de propuesta $\nabla_\theta \log \tilde{p}^r_t(\theta_t|x)$ y luego lo transforma en una estimación del puntaje verdadero $\nabla_\theta \log p_t(\theta_t|x)$ usando una identidad que relaciona ambos. Aunque evita los pesos de importancia, requiere la aproximación de varios términos de puntaje adicionales (marginales y condicionales) que no están inmediatamente disponibles, introduciendo complejidad y fuentes potenciales de error.
\end{itemize}

Las pruebas empíricas indican que los enfoques SNPSE-A, -B y -C, con sus respectivas correcciones, tienen un rendimiento significativamente inferior al de TSNPSE. Por lo tanto, TSNPSE emerge como el método secuencial preferido para la estimación del puntaje de la posterior, al ofrecer precisión y estabilidad superiores.


 


\end{multicols}

\bibliography{references}

\end{document}