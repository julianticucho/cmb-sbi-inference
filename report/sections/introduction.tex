\section{Introduction}
\subsection{Cosmological Context}
The discovery of the Cosmic Microwave Background (CMB) by Arno Penzias and Robert Wilson~\cite{penzias} in 1965 confirmed the existence of isotropic microwave radiation permeating the universe. Subsequent experiments showed that this radiation follows a blackbody spectrum with a temperature of \( T = 2.7255 \, \text{K} \) \cite{ryden}. The energy density of the CMB is low, with approximately 411 photons per cubic centimeter, and its characteristic wavelength (\(\sim 2\,\mathrm{mm}\)) places it in the microwave region of the electromagnetic spectrum~\cite{ryden}.

The existence of the CMB constituted key evidence in favor of the Big Bang model over the Steady State model. In an expanding universe that was initially hot and dense, matter was ionized, making the universe opaque. As the universe expanded and cooled, electrons and protons combined to form neutral atoms, allowing photons to decouple from matter. This radiation, which originally had a temperature of \(\sim 2970 \, \text{K}\), has cooled to the current \(2.7255 \, \text{K}\) due to cosmic expansion \cite{ryden}.

With the arrival of the \textit{WMAP} and \textit{Planck} probes, high-precision measurements of CMB anisotropies were achieved, which are fundamental for understanding the primordial inhomogeneities that gave rise to the large-scale structure of the universe. Combining these results with independent observations, such as type Ia supernovae and galaxy distribution, has allowed for refined estimates of cosmological parameters, consolidating the \(\Lambda\)CDM model as the standard framework of modern cosmology.

\subsection{Traditional Inference}
In modern cosmology, CMB analysis follows a well-established methodology, which now faces new challenges due to the volume and complexity of current data. The starting point is the temperature and polarization maps obtained by missions such as Planck~\cite{planck2018} and, in the near future, the Simons Observatory~\cite{simons2019}. These maps contain primordial information about both cosmological parameters and the initial conditions of the universe.

Traditional analysis takes these maps and extracts the power spectra \(C_{\ell}^{TT}\), \(C_{\ell}^{EE}\), \(C_{\ell}^{TE}\), and \(C_{\ell}^{BB}\), which quantify angular correlations at different scales (\(\ell \sim 180^\circ/\theta\)). This compression preserves Gaussian information and reduces data dimensionality. Mathematically, the power spectra correspond to the coefficients in the Legendre polynomial expansion of the two-point correlation function:

\begin{equation}
C(\theta)=\sum_{\ell=0}^{\infty}\frac{2\ell+1}{4\pi}C_{\ell}P_{\ell}(\cos \theta)
\end{equation}

where the two-point correlation function \(C(\theta)\) is defined as:

\begin{equation}
C(\theta)=\langle T(\hat{n}_1)T(\hat{n}_2) \rangle
\end{equation}

with \(T(\hat{n})\) being the temperature field in the unit direction \(\hat{n}\). The inference of cosmological parameters is carried out through likelihood functions, usually Gaussian, that compare observed spectra with theoretical predictions. Bayesian inference combines these likelihoods with prior distributions incorporating theoretical knowledge and independent constraints. In practice, exploration of the parameter space is typically performed using Markov Chain Monte Carlo (MCMC) methods, which sample the posterior distribution in high-dimensional spaces.

This approach has allowed precise determination of the \(\Lambda\)CDM model parameters but requires assuming a likelihood function that is not always exact. Moreover, the Gaussian likelihood assumption may be too restrictive. In the following section, an alternative simulation-based approach is presented, which allows inference in situations where the likelihood is intractable or unknown.

\subsection{Simulation-Based Inference}
The traditional approach presents two fundamental limitations: the loss of non-Gaussian information during compression to power spectra, and the dependence on Gaussian assumptions in the likelihood. Both problems are closely related, since attempting inference with non-Gaussian summary statistics while assuming a Gaussian likelihood can lead to biased results. Simulation-based inference (SBI) aims to perform statistical inference in situations where the likelihood function is intractable or unknown.

Hybrid methods such as ABC (\textit{Approximate Bayesian Computation}) address this problem by directly comparing simulated data with observed data using a distance metric. In this approach, multiple simulations are generated from different parameter values \(\theta\) and those for which the simulated data \(x\) sufficiently resemble the observed data, according to a tolerance threshold, are retained. This method strongly depends on the choice of summary statistics and may require a very large number of simulations to obtain a reasonable approximation of the posterior distribution.

More recently, modern machine learning-based techniques such as SNPE (\textit{Sequential Neural Posterior Estimation}) \cite{SNPE_C} or NPSE (\textit{Neural Posterior Score Estimation}) \cite{NPSE_1} \cite{NPSE_2} have revolutionized this approach. Instead of directly comparing data, these methods reformulate the problem as one of density estimation: the joint distribution of simulated pairs \((\theta, x)\) is modeled, allowing approximation of the posterior distribution \(p(\theta|x)\). Tools such as deep neural networks and normalizing flows enable training expressive models that learn directly the relationship between data and parameters from synthetic samples, avoiding the explicit computation of the likelihood and making inference much more efficient.

\subsection{Project Objective}
The main objective of this project is the implementation of simulation-based inference techniques to estimate cosmological parameters from angular power spectra of the CMB. In the methodology section, the problem is introduced and the proposed methodology is described, while the results section presents the outcomes obtained using the proposed approach.


